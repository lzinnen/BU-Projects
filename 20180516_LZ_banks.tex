\documentclass[english]{article}
\usepackage[T1]{fontenc}
\usepackage[latin9]{inputenc}
\usepackage{geometry}
\geometry{verbose,tmargin=1in,bmargin=1in,lmargin=1in,rmargin=1in}
\usepackage{textcomp}
\usepackage{amsmath}
\usepackage{babel}
\begin{document}
\begin{verbatim}
TO:        Robert G. King
             Stephen J. Terry
FROM:      Luke Zinnen
DATE:      17 May, 2018
SUBJECT:   Bank Mechanics: Me and GK
\end{verbatim}

\section{Overview}

This memo provides a description of the banker agents in my generalization of the Gertler-Kiyotaki (2015)
model. The underlying ideas and resultant mathematics are then contrasted with the bank in GK.

\section{My banks and bankers}
The bank is one of two agents in an economy. It purchases and sells a capital asset in a competitive 
market, and uses the capital it owns at the start of any period to produce a consumption good at 
an exogenously determined rate of $Z_t$ units per unit of capital. These capital holdings are funded internally 
by the bank's net worth (mark to market book value) $N_t$ and externally by deposits $D_t$ bearing interest rate 
$R_t$ (equivalently, sell $R_t*D_t$ face value discounted notes) to households.

The bank is indefinitely lived, and will continue in operation until it enters bankruptcy. The bank 
itself has assets and liabilities, but no intrinsic interests; it is run for the bankers who own and 
operate it.

In each period, there is a continuum of order 1 of bankers at the bank. Each period, a fraction 
$1-\sigma$ of these bankers retire, their interest in the bank bought out by the remaining fraction 
and a mass $1-sigma$ of new bankers who enter the firm. Along with investment and financing, this 
entry and exit of bankers determines the evolution of the bank's net worth:

\begin{equation}N_{t} = \sigma*\left(\left(Z_t + Q_t\right)*K^b_{t-1} - R_{t-1}*D_{t-1}\right) + \left(1 - \sigma\right)*w\end{equation}

where 

$N_t$ is the internal resources available to the bank for purchasing capital at time $t$

$\sigma$ is the exogenous fraction of bankers who enter the period and remain with the bank; likewise, $(1-\sigma)$ is the fraction who retire and join

$\left((Z_t + Q_t)*K^b_{t-1} - R_{t-1}*D_{t-1}\right)$ is the value of firm assets entering the period:
the current production generated by $(Z_t)$ and the sale price of $(Q_t)$ capital held, less debt repayments. A fraction 
$(1-\sigma)$ of this is paid out to retiring bankers

$(1 - \sigma)*w=W$ is equity paid in to the bank by entering bankers; each of the $(1-\sigma)$ new entrants
arrives with and contributes an endowment $w$ of the consumption good. For this, they receive an
equity stake in the bank equal to the fraction of the bank's book value attributable to their paid-in 
capital: $\frac{(1 - \sigma)*w}{N}$. Continuing bankers, those 
who entered the period as paid-in owners of the bank and did not retire, then own a fraction 
$\frac{N - (1-\sigma)*w}{N}$ of the bank.

Having resolved the bank's transactions with its owners, it must determine its market actions for the 
period: how much debt to issue, and how much capital to hold. The bank do so by maximizing its value 
as a going concern subject to two constraints: a balance sheet constraint and an incentive compatibility 
constraint.

The going-concern value of the bank (that is, its value) is the discounted expected value of owning the bank going into the following 
period. This is composed of two parts: payouts to retiring bankers (out of resources held by the bank 
before payouts \emph{or receiving capital from new bankers}: $\frac{N_t - (1-\sigma)*w)}{\sigma}$) 
and the ownership stake in the bank left over after retirements and new bankers joining (and thus 
diluting existing bankers' share of ownership). This value reflects bankers' rate of time preferenct 
$\beta$ and their indifference to the timing of their consumption.

\begin{equation}V_{t} =  \frac{1-\sigma}{1 - \sigma}\frac{1-\sigma}{\sigma}(N_t - (1-\sigma)*w) + \beta*\frac{\sigma}{\sigma}\frac{N_t - (1-\sigma)*w}{N_t}E\left[V_{t+1}\right]                                                        \end{equation}

Consider normalizing to 1 share. Then the value derives from: a $1-\sigma$ probability of receiving a $\frac{1}{1-\sigma}$ share 
of a payout amounting to $\frac{1-\sigma}{\sigma}(N_t - (1-\sigma)*w)$; a $\sigma$ probability of 
owning a $\frac{1}{\sigma}$ share of the equity held by continuing bankers, which is itself a fraction 
$\frac{N_t - (1 - \sigma)*w}{N_t}$ of a bank with continuation value $\beta*E\left[V_{t+1}\right]$.

Note that the entry of new bankers has something of an amplifying effect on the continuation value 
for existing bankers: if a bank ends up with few resources entering a period, then the degree of 
dilution faced by continuing bankers is relatively great; if it enters a period with many resources, 
then not only is the bank itself likely to be worth more, but the portion of ownership which remains 
in the hands of continuing bankers is closer to 1.

We can thus see that for any variables external to the bank (capital price, productivity, or interest rate, 
which are restricted to be nonnegative) the bank's net worth and going-concern value are increasing 
in capital held and decreasing in deposits. Unconstrained (and with a well-behaved price structure, 
that is, one where the price of capital does not rapidly fall as bank holdings increase or bank debt 
falls), the bank would choose to hold the economy's entire capital stock and issue no debt.

They are prevented from doing so by 2 constraints. The first is the balance sheet constraint:

\begin{equation}Q_{t}*K^b_t \leq N_t + D_t \end{equation}

That is, the market value of capital held must be no greater than the bank's book value plus the debt 
it issues. (Alternatively, the proceeds from discount notes sold.) If capital and debt are continuous, 
this will hold with equality if the price structure is well behaved.

The second is an incentive compatibility constraint. The bankers can collectively choose to issue 
deposits and purchase capital, and then divert a fraction of the asset holdings for their own benefit 
(in which case the bank will then cease operation), 
selling them in the market and receiving a fraction of the proceeds $\theta$, reflecting excess  
costs of operating clandestinely. Depositors are aware of this possibility, and thus will only deposit 
funds up to the point the bankers are indifferent between holding the bank as a going concern and 
liquidating it and absconding with the proceeds: the fraction of the market value of assets held by 
the bank which can be successfully diverted must be less than or equal to the discounted expected 
value of the bank next period.

\begin{equation}Q_{t}*K^b_t \leq \beta*E\left[V_{t+1}\right] \end{equation}

The bank 

These constraints limit the available combinations of capital and deposits available to the bank. 
Together, they require that the average unit of capital purchased not only have sufficiently positive 
expected returns to cover debt repayments, but to leave enough surplus to maintain depositors' 
confidence even before the next period's productivity level is realized and a bank run sunspot can 
occur.

Banks are considered to cease operation only in the event of bankruptcy. Bankruptcy occurs when 
depositors collectively do not roll over the bank's debt, leaving it unable to fund capital purchases. 
This can occur if, conditional on deposit flight, the price of capital will be sufficiently low that 
the bank will have negative net worth, that is, will not have the resources to meet maturing obligations 
to depositors. If a run occurs, existing bankers will never consume; either they exit with no net worth, 
or they remain as an accounting fiction for a new bank that will arise, but have no ownership stake 
therein, existing just to keep a unit mass of bankers. Bankers entering the period of a bank run 
either immediately consume their endowment or wait for the following period to form a new bank, 
along with those bankers who ender that period. If they wait, a fraction $(1-\sigma)$ of the run-period 
entrants exit before forming the bank the next period. 

Given its feasibility, a run occurs with a probability linked to the financial state of the bank. A 
run is considered inevitable if the bank has negative net worth even in the absence of a run. Otherwise, 
run probability is an increasing function of depositor losses on deposits in the event of a run, and 
can be realized based on an underlying Markov sunspot process or one that is IID.

Using knowledge about future prices and run probabilities conditional on bank asset and deposit levels 
and the productivity process, the bank chooses one best combination $(K^b_t,  D_t)$ from the universe 
of feasible combinations: the one that maximizes $E[V_{t+1}]$ subject to the constraints. The period 
ends, a new productivity shock is realized, and if a run is feasible, the value of the sunspot is 
realized and observed, beginning the equity and investment activities for the new period.



\section{Comparison to those of GK 2015}
The most significant difference between the banks in my proposed economy and those of GK2015 is that 
mine are multi-member firms, and theirs are a continuum of independent operations. Rather than a single 
bank with members entering and exiting, banks operate in parallel, exiting bankers consume the net worth 
of the bank they run, continuing bankers retain all earnings, and new bankers start new banks. 

Equation 1 thus has a bank-by-bank analog in the original GK model:

\begin{equation}n_{t} = \left(\left(Z_t + Q_t\right)*k^b_{t-1} - R_{t-1}*d_{t-1}\right)\end{equation}

This is an exact analog to the resource process of a bank in my model, before payouts and capital 
being paid in. Upon aggregation, integrating over all bankers in the GK model (apart from scale, all 
banks in the GK model are identical), Equation 1 is recovered exactly. 

A banker's value function in GK2015 is, as a result of their sole proprietorship, considerably simpler 
than Equation 2. It is:

\begin{equation}v_{t} =  \left(1-\sigma\right)(n_t ) + \beta\sigma*E\left[v_{t+1}\right]                       \end{equation}

Three key differences appear between the value functions. The first is that in the original GK2015 version, 
the probability of entrance or exit remains after simplification; in my economy, the probabilities are 
exactly washed out by share counts: the probability of exit is divided by the count of exiters over 
whom the payout is split; the probability of remaining is divided by the count of remainers splitting 
ownership of the portion of the bank not owned by joiners. The second is that my bankers account for 
ownership dilution by entering bankers, which cannot be a feature when each banker owns their own 
bank. Lastly there is the distinction in payouts between the full value of the firm and a fraction of 
pre-equity transactions book value of the firm corresponding to the fraction of bankers exiting. 

These effects are generally offsetting. On the payout term, the new $(1-\sigma)$ terms in the denominator 
(splitting the payment between leaving bankers) is canceled by the one in the numerator (the size of 
the payment is proportional to the number of leavers), leaving the $\sigma$ in the denominator (reflecting 
that payments are made on an asset base before payouts are counted) increasing and the $(1-\sigma)w$ 
(the net worth relevant to payouts doesn't include newly paid in capital) decreasing the value of 
potentially being bought out in a period. Likewise, dilution of ownership of the going concern by new 
bankers is offset by the buyouts of exiting ones. The exact net difference will be state dependent. 

Additionally, an individual banker would take all other banks' actions as given, and their own decisions 
as affecting nothing but their own probability of default (as it is a function of their depositors' 
recovery ratio); in contrast, a unified bank must take into account the effects its investment and 
debt levels have on the aggregate state. Thus, my model more closely resembles the decision-making 
process of a large bank which can potentially move markets itself than the infinitesimal banks of the 
GK model. 

However, it should be noted that Equation 2 directly corresponds to the aggregate value of the banking 
system owned by old bankers on entering a period in the GK 2015 model. 

The constraints, apart from the scale difference between an independent banker and a banking firm, 
are identical. 

\section{Motivation for changes}
I chose to move to a multi-banker firm in order to support heterogeneous banking with adjustment costs 
in the model. Adjustment costs at the bank level would tend to introduce differences in behavior 
based on scale; because GK 2015 relied on constant returns to scale to generate a representative bank, 
there would no longer be a representative bank identical to all others up to a factor 
of net worth. If entering bankers form new banks each period, this would necessitate tracking each 
banker cohort separately, even though they face the same sequence of productivity levels, asset prices, 
and interest rates. 

By having a single bank in a closed economy, the feasibility of having an economy of many households 
and banking systems, facing different shocks but trading in one or more overlapping markets, increases 
greatly. 
\end{document}
