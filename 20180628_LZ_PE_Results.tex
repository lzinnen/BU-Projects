\documentclass[english]{article}
\usepackage[T1]{fontenc}
\usepackage[latin9]{inputenc}
\usepackage{geometry}
\usepackage{graphicx}
\geometry{verbose,tmargin=1in,bmargin=1in,lmargin=1in,rmargin=1in}
\usepackage{textcomp}
\usepackage{amsmath}
\usepackage{babel}
\usepackage{float}
\begin{document}
\begin{verbatim}
TO:        Robert G. King
             Stephen J. Terry
FROM:      Luke Zinnen
DATE:      28 June, 2018
SUBJECT:  Partial Equilibrium Testing: Results
\end{verbatim}

\section{Overview}

This memo provides an overview of the results of looking at the intermediate output of my solution method 
for the recursive GK model, and what these results imply for future attempts to address the solution's 
non-convergence.

\section{Setup}
The primary input to the test case was the intermediate result of the solution method after some 
thousands of iterations of the outer loop (the "price" portion of the "price-value/policy function 
iteration"). The relavant inputs included the asset price, interest rate on deposits, bank net worth, 
bank run probability, and household consumption. These were used as constant inputs to the inner loop 
(determining the bank's policy function given exogenous prices); prices, interest rates, bank net worth, 
and household consumption were updated using the resulting bank policy function. These new values and
the ergotic distribution were the final variables of interest.

This was repeated varying the input values uniformly up or down in the relavant time period by 10 basis points, and results 
were compared to the baseline. This was done for both inner loops based on both value function iteration 
and policy function iteration, though discussion will focus primarily on value function iteration, 
for reasons to be explained in the following section.



\section{Results}
\subsection{Baseline Tests}
The most immediately striking result of the baseline test is the distinction between results for using 
value or policy function iteration for the inner loop. The policy functions differed for several states, 
and those states had a difference in value function approximately ten times as great as the difference 
between the value and policy results for other states. The policy function iteration had the bank in 
those states unable to hold capital, while the policy resulting from value function iteration involved 
positive capital holdings. 

Although the difference in ergotic distribution at those specific states was small (2.32E-34), other 
states had considerable difference between the resultant distributions for the two methods: of order 
1E-4, or more than the average density over all states.

This suggests that a focus on stability between price updates is likely to be important for obtaining 
correct and consistent results, and the known convergence and superior stability properties of value 
function iteration (and importantly less apparent tendency to get stuck in bad corner policies) make it 
or similar methods preferred.

\subsection{Adjusting the Risky Rate Paid on Deposits}
On decreasing or increasing the rate banks must pay on deposits, large and consistent changes to the 
bank's value function result, which are particularly visible when seen as a ratio against the baseline 
result.  There is a discrete break in the changes following a roughly straight line in bank capital-bank debt 
space from the origin northeast at approximately a slope of 0.75: this ridge is where the bank's original 
ergotic distribution was largely concentrated, due to being the region operating the most capital without 
violating one of its constraints. Reducing the rate paid on deposits increases the states' values immensely 
(coming from a very low base in the low capital, high debt side), but has comparitively minor effect where 
the balance sheet constraint was already slack. The reverse is true on increasing R: there is no effect 
where the balance sheet constraint bound in the baseline, and there was a large reduction in value 
espeically along the margin of the ridge where previously the balance sheet constraint was close to 
binding, and the reduction gets smaller you consider states further removed from this cutoff line.

\begin{figure}[H]
\centering
		\includegraphics[width=.5\textwidth]{ergVbase.png}
\end{figure}

\begin{figure}[H]
\centering
		\includegraphics[width=.5\textwidth]{VvalFracVvalDR.png}
\end{figure}
\begin{figure}[H]
\centering
		\includegraphics[width=.5\textwidth]{VvalFracVvalIR.png}
\end{figure}


There are many and large changes in capital choice with a small shift in the risky rate. These are not 
concentrated solely along the active ridge with high ergotic density, and are far from uniform in direction:
many states see higher capital use by banks when rates rise, and the reverse, though on average the 
direction is at least correct. 

\begin{figure}[H]
\centering
		\includegraphics[width=.5\textwidth]{VKbpLessVKbpDR.png}
\end{figure}
\begin{figure}[H]
\centering
		\includegraphics[width=.5\textwidth]{VKbpLessVKbpIR.png}
\end{figure}

A decrease in R brings a volatile change in debt positions: very jumps generally about 0.4, surrounded by states with no change. Similar results for an increase in R, reversed and with only very small (0.001) magnitude.

\begin{figure}[H]
\centering
		\includegraphics[width=.5\textwidth]{VRDpLessVRDpDR3.png}
\end{figure}
\begin{figure}[H]
\centering
		\includegraphics[width=.5\textwidth]{VRDpLessVRDpIR2.png}
\end{figure}

Decreasing the interest rate tends to raise the risky asset's price, in a way that's most pronounced just 
to the high debt/low capital side of the ergotic ridge, by about 0.06-0.2 depending on productivity level. 
Prices rise less and less frequently on the opposite side, especially far from the ridge. Raising the interest 
rate affects mainly the region to the good side of the ridge, but in a fairly volatile way with some large increases 
in price but many (if somewhat smaller) decreases nearby.

\begin{figure}[H]
\centering
		\includegraphics[width=.5\textwidth]{VQnewLessVQnewDR3.png}
\end{figure}
\begin{figure}[H]
\centering
		\includegraphics[width=.5\textwidth]{VQnewLessVQnewIR2.png}
\end{figure}

The response of the output estimate of R to a reduction in the input estimate is a consistent and 
large (-0.1) fall near the ridge on the bad side, which rapidly reduces to a value closer to -0.01. On the 
opposite side there is some volatility, but little change on average. A rise in the input R retains the 
output volatility on the good side of the ridge, but has almost no effect on the low bank net worth side.

\begin{figure}[H]
\centering
		\includegraphics[width=.5\textwidth]{VRnewLessVRnewDR3.png}
\end{figure}
\begin{figure}[H]
\centering
		\includegraphics[width=.5\textwidth]{VRnewLessVRnewIR3.png}
\end{figure}

Reducing the interest rate increases bank net worth along the ridge by multiples into the 20s, but has 
minimal effect elsewhere. Increasing the interest rate can, interestingly, increase output bank net worth 
estimates, by a fair amount (as high as -3x), but is in general more mixed but not especially volatile. This takes 
place on the high net worth side of the ridge only, though; no changes are observed on the low net worth side.

\subsection{Adjusting Bank Survival Probability}
Reducing the probability of a bank surviving has a minor impact on estimated state values in the low net worth 
region, on the order of a 1\% reduction. This is larger on the opposide side of the ridge, with the magnitude 
depending on the productivity level, with higher productivity levels more responsive. Besides these general 
regions, on the ergotic ridge itself there are cases of significantly larger falls in value. Increasing the survival 
probability has little effect, of order at most 1\% and generally much less, concentrated near and on the active
ridge.

\begin{figure}[H]
\centering
		\includegraphics[width=.5\textwidth]{VvalFracVvalDpiL.png}
\end{figure}
\begin{figure}[H]
\centering
		\includegraphics[width=.5\textwidth]{VvalFracVvalIpiL.png}
\end{figure}

Apart from reducing survival probability with the lowest productivity level, very few states 
see a change in capital chosen by banks. For that case, the changes are concentrated 
close to and parallel to but not on the ergotic ridge, on the high net worth side. The location
of states which do have changes does not have much pattern. The same generally holds for debt choice.

\begin{figure}[H]
\centering
		\includegraphics[width=.5\textwidth]{VKbpLessVKbpDpiL.png}
\end{figure}
\begin{figure}[H]
\centering
		\includegraphics[width=.5\textwidth]{VvalFracVvalIpiL.png}
\end{figure}
\begin{figure}[H]
\centering
		\includegraphics[width=.5\textwidth]{VRDpLessVRDpDpiL.png}
\end{figure}
\begin{figure}[H]
\centering
		\includegraphics[width=.5\textwidth]{VRDpLessVRDpIpiL.png}
\end{figure}

Decreasing bank survival probability predictably decreses the ergotic distribution along 
the region of non-default states with positive initial density, but also increases non-default 
density at low (or 0) levels of debt issued and capital held; more time is spent in the early 
recovery from default. For corresponding bank run states, similar holds in reverse, but with much lower 
absolute levels of change and few cases of reduced density. 

Increasing bank survival probability has smaller overall effects, generally in the opposite 
direction (increasing density for survival states), and with instances of reduced density 
in survival states being balanced by increases at nearby points on the ridge rather than 
near the origin. Bank run states are affected to an even smaller degree than survival states, 
and where there are increases in density tend to be at very aggressive capital levels.

\begin{figure}[H]
\centering
		\includegraphics[width=.5\textwidth]{VergLessVergDpiL.png}
\end{figure}
\begin{figure}[H]
\centering
		\includegraphics[width=.5\textwidth]{VergLessVergDpiLbr.png}
\end{figure}
\begin{figure}[H]
\centering
		\includegraphics[width=.5\textwidth]{VergLessVergIpiL.png}
\end{figure}
\begin{figure}[H]
\centering
		\includegraphics[width=.5\textwidth]{VergLessVergIpiLbr.png}
\end{figure}

Decreasing survival probability reduces the price of the risky asset by 
up to about 3\%, but in few states. For the medium high productivity level, 
it decreases the price consistently in the high net worth region, but by only a few 
basis points. With an increase in survival probability, there is rather a mix in some 
few states of similar magnitude, and a similar consistent fall in the price in the 
high net worth region but for low productivity levels.
VQnewLessVQnewDpi

\begin{figure}[H]
\centering
		\includegraphics[width=.5\textwidth]{VQnewLessVQnewDpi.png}
\end{figure}
\begin{figure}[H]
\centering
		\includegraphics[width=.5\textwidth]{VQnewLessVQnewDpi2.png}
\end{figure}
\begin{figure}[H]
\centering
		\includegraphics[width=.5\textwidth]{VQnewLessVQnewIpi.png}
\end{figure}

The effect of decreasing survivial probability is generally to increase rates, by 
up to a few percentage points, but again in relatively few states. Mirroring the 
effect on price, rates increase consistently but by only a few basis points in 
the medium-high productivity state. 

Increasing the survival probability acts similarly, either pushing the rate up or down 
a few percentage points or consistently increasing it by a few basis points in the 
high net worth area depending on productivity level.

\begin{figure}[H]
\centering
		\includegraphics[width=.5\textwidth]{VRnewLessVRnewDpi.png}
\end{figure}
\begin{figure}[H]
\centering
		\includegraphics[width=.5\textwidth]{VRnewLessVRnewIpi.png}
\end{figure}

Increasing or decreasing survival probability has similar effects on bank net worth 
as it does on price and interest rates. 

\begin{figure}[H]
\centering
		\includegraphics[width=.5\textwidth]{VNnewLessVNnewDpiL.png}
\end{figure}
\begin{figure}[H]
\centering
		\includegraphics[width=.5\textwidth]{VRnewLessVRnewIpi.png}
\end{figure}


\subsection{Adjusting Capital Price}
Values are extremely responsive to changes in capital price, ranging from a few percent to 
single digits in the high net worth region, to a ratio change in the hundreds close to 
the ridge on the bad side, to thousands or even billions in cases where there was a 
very low base in the low net worth region for some productivity levels.

Increasing the capital price has very different results depending on productivity level. 
In all cases the value change is flat or nearly so in each region, but the levels (except for 
the highest productivity level) are very different, reducing value to 0 in the low net worth region 
for low to medium productivity, and leaving the value in the high productivity region 
largely unchanged. High productivity levels see no change in the 
high net worth region, but smaller falls (about 25\% for the medium high level, or none 
for the high productivity level) for the low net worth region, and spikes lower along the ridge.


\begin{figure}[H]
\centering
		\includegraphics[width=.5\textwidth]{VvalFracVvalDQ.png}
\end{figure}
\begin{figure}[H]
\centering
		\includegraphics[width=.5\textwidth]{VvalFracVvalIQ.png}
\end{figure}
\begin{figure}[H]
\centering
		\includegraphics[width=.5\textwidth]{VvalFracVvalIQ2.png}
\end{figure}

Decreasing the capital price has volatile effects, with large swings in either direction 
close to each other both in the low net worth region and the very high net worth region. 
Changes in both directions exist but of lower magnitude in the part of the high net worth 
region relatively close to the ridge. Changes with increasing Q are less pronounced and 
follow less of a clear pattern, but crucially include the capital choice at the origin fallinng to 0; 
the origin is thus a fixed point and any possibility of reaching a bank run state will result in 
a degenerate distribution at the origin. Debt choices are more or less similar.


\begin{figure}[H]
\centering
		\includegraphics[width=.5\textwidth]{VKbpLessVKbpDQ.png}
\end{figure}
\begin{figure}[H]
\centering
		\includegraphics[width=.5\textwidth]{VKbpLessVKbpIQ.png}
\end{figure}
\begin{figure}[H]
\centering
		\includegraphics[width=.5\textwidth]{VRDpLessVRDpDQ.png}
\end{figure}
\begin{figure}[H]
\centering
		\includegraphics[width=.5\textwidth]{VRDpLessVRDpIQ.png}
\end{figure}



Decreasing the asset price leads to mixed effects for non bank run states, roughly 
seeing more density in high capital states and less in low, and slightly shifting the ridge
in the low debt direction. Similar with much lower absolute magnitudes holds for the 
bank run case, and with fewer falls in density.

For an increase in Q, all densities fall to 0 except for the origin.


\begin{figure}[H]
\centering
		\includegraphics[width=.5\textwidth]{VergLessVergDQ.png}
\end{figure}
\begin{figure}[H]
\centering
		\includegraphics[width=.5\textwidth]{VergLessVergDQbr.png}
\end{figure}
%\begin{figure}[H]
%\centering
%		\includegraphics[width=.5\textwidth]{VergLessVergIQ.png}
%\end{figure}

Decrease in input Q leads to large changes concentrated near the ridge. Particularly large
increases in estimated Q over 0.1 are found just on the low net worth side of the ridge, while smaller changes
are found on the high net worth side and some large falls right on the ridge, particularly near the origin.

Increases in input Q have no effect on output estimated Q on the low net worth side 
of the ridge, but fairly volitile if lower magnitude (about 0.04) changes in both directions 
on the high side. Repeating the test with smaller values of deviations from baseline values, 
down to at least a tenth of a basis point, still resulted in very large fractions of all states 
with the bank forced to choose the no-capital no-debt policy, including at the origin, resulting in 
a degenerate distribution.




\begin{figure}[H]
\centering
		\includegraphics[width=.5\textwidth]{VQnewLessVQnewDQ.png}
\end{figure}
\begin{figure}[H]
\centering
		\includegraphics[width=.5\textwidth]{VQnewLessVQnewIQ.png}
\end{figure}


Reducing Q leads to reductions in the risky rate over most of the low net worth region, most pronounced
near the ridge and with low capital. Along the ridge near the origin there is substantial increase in R, and 
otherwise throughout the high net worth side there is on average little to no change, but a fair amount of 
states increasing or decreasing by up to a couple of percentage points. With an increase in asset price, 
there is more volatility on the high net worth side, but the increase along the ridge is no longer present and 
on the low net worth side there is substantially no change in interest rate levels.



\begin{figure}[H]
\centering
		\includegraphics[width=.5\textwidth]{VRnewLessVRnewDQ.png}
\end{figure}
\begin{figure}[H]
\centering
		\includegraphics[width=.5\textwidth]{VRnewLessVRnewIQ.png}
\end{figure}

Decreasing the asset price has no effect on the new estimate of net worth of banks on the low net worth side of the ridge, 
but leads to considerable volatility on the high net worth side, with the greatest amount closest to the ergotic ridge. 
Closest to the ridge there tend to be increases, and just further away large decreases, and in the rest of the 
high net worth region a rapidly shifting mix of the two but of lesser magnitude. Increasing Q likewise has no effect 
on the low net worth side of the ridge, but volatility on the high side and expecially where capital is very high, sometimes 
with a series of large increases along the ridge itself (depending on productivity level).


\begin{figure}[H]
\centering
		\includegraphics[width=.5\textwidth]{VNnewLessVNnewDQ.png}
\end{figure}
\begin{figure}[H]
\centering
		\includegraphics[width=.5\textwidth]{VNnewLessVNnewIQ.png}
\end{figure}


\section{Takeaways}
I believe that there are two key features of the results shown. The first is the large and rapid changes in responce 
on a state by state basis, notably including a great deal of volatility within regions of fairly flat average responses 
to parameter changes. The second is that responses are often most pronounced at or near the region--the 
states running in a band roughly from the origin to (bank capital = 1.0, bank debt = 0.6) at each productivity 
level--where the bank's policy function had it most frequently.

The latter fact is crucial in that it demonstrates that the observed non-convergence in outer-loop 
variables reported in previous memos is not a mere artifact of repeated changes in estimates for states 
which are never reached and are thus largely irrelevant, in the sense that they only need to be estimated 
precisely enough to confirm they are not viable choices in the bank's menu of policy functions. Were it 
the case that the states near those the bank chooses to occupy were largely stable to small parameter 
changes, even as significant moves occurred far from the region of efficient choices (but not large enough 
to render them plausible alternatives), then I might be able to justify focusing on the well-estimated relevant 
portion of the state space, but that is not the case.

The former fact, the volatility within regions, suggests two major conclusions to me. The first, and more 
certain, is that it will be necessary to very correctly estimate each new set of outer-loop variables 
conditioned on the future. If (admittedly more extreme, in the sense of all states sharing the same change) 
small perterbations of parameter values, 10bp in either direction in each case, or 1 to 2 orders of magnitude 
smaller than the largest period-by-period estimate updates and of similar order to mean absolute changes, 
are capable of resulting in such large changes in update estimates, then finding at least the conditionally 
correct policy function with as great precision as possible before returning to the outer loop to update 
asset prices, interest rates, etc. is likely a necessity. This precludes the use of various solution methods 
I have used so far for either the full model or certain iterations of the simplified model with an alternative 
investment technology taking the place of the banking system. These include both policy function 
iteration and value function iteration to solve the inner loop, as well as simple alternating between 
a single step of the value function on the inner loop and updating outer-loop variables. I believe the 
greatest stability is achievable through the following: taking one period ahead values as given and 
an initial estimate of current period values (which may be the same), solve a one-period maximization 
problem for the bank and use the result to update estimates of current period values. Repeat this 
process holding the future values fixed but with the new estimates of current period values. Only 
when this repetition converges on an updated estimate of current period values should the estimate 
be held fixed and the process move backwards a time step, using the converged estimate of current 
period values as the new future values. While this may prove helpful in increasing accuracy and 
precision of estimates, the inability to completely reset the bank's value function each period will 
tend to increase the number of periods before which initial guesses of the value function are attenuated.
Additionally, I have used this method in some cases with variations of the 
stripped down model, and though it will generally converge to new estimates, the time involved is 
often extensive.

The second conclusion I draw from the rapid changes in new estimates throughout the state space, though 
less strongly indicated, is that the marginal value of increasing the fineness of the grid is likely still 
high, and yield meaningfully improved smoothness. Experience with variations of the simplified model 
suggests that increasing the number of grid points results in a less than proportional reduction of 
estimate update sizes, such that expansion without other changes in method is not likely to be sufficient 
to reduce instability to an acceptable level on its own (while remaining within time and memory constraints), 
but it may be a useful means of getting some improvement.


\end{document}                          