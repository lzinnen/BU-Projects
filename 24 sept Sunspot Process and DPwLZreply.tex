%% LyX 2.1.4 created this file.  For more info, see http://www.lyx.org/.
%% Do not edit unless you really know what you are doing.


\documentclass[english]{article}
%%%%%%%%%%%%%%%%%%%%%%%%%%%%%%%%%%%%%%%%%%%%%%%%%%%%%%%%%%%%%%%%%%%%%%%%%%%%%%%%%%%%%%%%%%%%%%%%%%%%%%%%%%%%%%%%%%%%%%%%%%%%%%%%%%%%%%%%%%%%%%%%%%%%%%%%%%%%%%%%%%%%%%%%%%%%%%%%%%%%%%%%%%%%%%%%%%%%%%%%%%%%%%%%%%%%%%%%%%%%%%%%%%%%%%%%%%%%%%%%%%%%%%%%%%%%
\usepackage[T1]{fontenc}
\usepackage[latin9]{inputenc}
\usepackage{geometry}
\usepackage{textcomp}
\usepackage{amsmath}
\usepackage{babel}

\setcounter{MaxMatrixCols}{10}
%TCIDATA{OutputFilter=LATEX.DLL}
%TCIDATA{Version=5.50.0.2960}
%TCIDATA{<META NAME="SaveForMode" CONTENT="1">}
%TCIDATA{BibliographyScheme=Manual}
%TCIDATA{LastRevised=Tuesday, September 26, 2017 17:36:57}
%TCIDATA{<META NAME="GraphicsSave" CONTENT="32">}

\geometry{verbose,tmargin=1in,bmargin=1in,lmargin=1in,rmargin=1in}

\input{tcilatex}

\begin{document}

\begin{verbatim}
TO:        Robert G. King
FROM:      Luke Zinnen
DATE:      September 25 2017
SUBJECT:   Sunspot process and dynamic programming
\end{verbatim}

\section*{Bank run sunspots and DP formulation}

In determining the behavior of the bank run/financial accelerator model it
is necessary to define a process for assigning what banks are subject to a
run in each period, and incorporating that process into the dynamic
programming problems the bankers and households solve.

I propose three potential methods to handle this addition, which may roughly
be distinguished by the combination of aggregate and indiosyncratic
components to the sunspot shocks.

Fundamental productivity shocks are considered to be handled identically
across the methods that follow.

\subsection{Purely Idiosyncratic}

In this formulation, a bank-household pair has a fundamental state vector
(Kb, RD, g, Z) of the quantity of capital operated by banks, the deposits
owed to the household (inclusive of interest), the household's holdings of
the risk free asset,\textbf{\ and local and aggregate productivity: single
state?}. \textit{No, or not exactly; that was an implicit merging of the two
notationally, but with independent driving processes. Knowing the resulting
composite level of productivity would not, in general, be sufficient to know
the distribution of the composite in future periods. 
Keeping it as a single term was meant (in the paper and here) to keep the
notation a little more compact.}

Given the aggregate state, a recovery ratio $X^{i}\in \lbrack 0,1]$ exists
in the event of a run. \ \textbf{The sunspot?} \textit{Strictly speaking, in
all states regardless of the outcome, it's just uninteresting most of the time, 
where X=1. X<1 if: there is a fundamental bank failure or if the sunspot 
occurs.} $\xi ^{i}\sim U(0,1)$ is
drawn, and if $X^{i}<\xi ^{i}$ the bank experiences a run. Additionally, if
any pairs exist for which, though the recovery ratio is positive, the bank
cannot repay deposits owed even in the absence of a run sunspot (bank
failure is fundamental), that pair experiences a run. \ \textbf{It will be
interesting to see how frequent it is, as it might not be an equilibrium
outcome depending on the risk tolerances of various agents and the range of
fundamental shocks. } \textit{As a fundamental I would expect it to be fairly
rare unless the underlying productivity processes have high variance and/or
persistence. And even then, that should induce risk-limiting behaviors.}

Thus the states are fully defined by (Kb, RD, g, Z, br), and agents can
optimize their choices of Kb, RD, and g subject to the known current state
and the productivity transition process, as well as the probability of a
bank run occuring in each (Kb, RD, g, Z) combination.

This method generates a degree of symmetry across fundamental states, in
that for all states where a run could occur some pairs will experience one,
and the fraction that do is decreasing in the fundamental strength of the
banks. \textbf{Are you using some form of LLN\ reasoning here? }%
\textit{LLN in assuming a continuum of pairs, and some ceteris paribus
in that it's more directly about choosing potential leverage levels given the same 
productivity states rather than actual leverage given an entering state. 
Higher expected returns on the asset lead to higher leverage and, other
things equal, lower risk of default, so that muddies the issue over all states.}
Additionally, it has the benefit that (assuming the idiosyncratic shock is
IID, which I find desirable because it keeps more focus on stresses arising
from fundamentals) the state space is increased as little as possible
relative to GK. However, this comes at the cost of there never actually
being (for a large number of pairs) larger or smaller runs, given a
fundamental state, defined as a full distribution of endogenous state
variables and productivities.

\subsection{Purely Aggregate}

The same fundamental state vector applies when the sunspot is purely
aggregate. However, an economywide bank run sunspot $\xi^{agg}$ is added,
which is necessary for determining the recovery ratio. $X^{i}\in[0,1]$ is
now a function of (Kb, RD, g, Z, $\xi^{agg}$). Given this state, a bank is
similarly subject to a run if $X^{i}<$$\xi^{agg}$, as well as if in the
state a failure is fundamental.

Thus the states are fully defined by (Kb, RD, g, Z, $\xi^{agg}$), though for
computational purposes finding an equilibrium will require a bank run/no-run
state as well, and agents can optimize their choices of Kb, RD, and g
subject to the known current state (Z, $\xi^{agg}$) and the productivity and
sunspot transition processes with known transition probabilities.

This method has the ability to generate different combinations of bank runs
depending on the sunspot that arises. As a result it can also demonstrate
the potential for nonlinear responses to sunspots of varying severities, as
an increasingly severe sunspot can have a direct effect of pushing some
banks into the run region, and pull in more due to effects on the price of
the risk free asset and capital. The downsides are the addition of a new
state variable (assuming $\xi ^{agg}$ to be Markov rather than IID, as this
must be more directly and precisely observable by agents in the economy and
would do less to drive cross sectional variation independent of fundamentals
than would be the case with a Markov idiosyncratic sunspot shock), and that
all banks of a given fundamental strength will stand or fall together, which
may be counterfactual and in any case eliminates confidence in individual
banks as a factor.

\textbf{I think it is fine to make }$\xi ^{agg}$\textbf{\ into a Markov
process, but do not really see why having an aggregate shock requires Markov
dynamics or why these could not have been introduced in the setup above. \
It certainly is one way to have "expected bank runs."}

\textit{I agree. I think Markov is a good option for persistence in household
confidence, but there's no reason a different discrete process, or even IID,
couldn't be used}

\subsection{Combination Aggregate and Idiosyncratic}

The states are the same as in the purely aggregate case, but an
idiosyncratic shock $\xi ^{i}$ is once again present (for example as a
zero-mean triangular or normal distribution). While the recovery ratio $%
X^{i}\in \lbrack 0,1]$ is still a function of (Kb, RD, g, Z, $\xi ^{agg}$),
banks now run if $X^{i}<$$\xi ^{agg}$+$\xi ^{i}$ (if neither survival nor
failure is a fundamental outcome). \ Don't \ understand this last statement.

Agents' optimization is similar to the purely aggregate method, but they now
incorporate probabilities of both bank failure and survival in each
(fundamentals, aggregate sunspot) state, rather than knowing with certainty
one outcome or the other would obtain.

This method offers the richest set of outcomes, allowing for broad or narrow
runs with the same fundamental state vector, while at the same time allowing
for similarly situated banks to fail or survive depending on differences in
local expectations. As in both of the previous cases, the \textbf{%
probability of bank failure} decreases as the fundamentals of the bank
become stronger. \textbf{As you move forward with this work, I think that it
is useful to think about the forecasting implications of the various models
that you are developing and plan to develop. \ So you would want to be
precsie about the conditioning infromation and even the sort of regressions
that \ might be used for predicting bank failure.} The same nonlinear run
effect as in the purely aggregate case would be expected to obtain. The
state space is effectively no larger than the purely aggregate sunspot case
given the same assumptions as above about the respective sunspot shocks.

\section*{Dynamic programming problems (combination)}

Letting that ratio $\frac{V_{t}}{n_{t}}\equiv\psi_{t}$, the bank's problem
becomes

\begin{equation*}
\begin{split}
\psi_{t}^{i}=\max_{\phi_{t}^{i}}E_{t}\left\{
\beta\left(1-\sigma+\sigma\psi_{t+1}^{i}\right)\left[%
(R_{t+1}^{b,i}-R_{t+1}^{i})\phi_{t}^{i}+R_{t+1}^{i}\right]\right\} \\
=\max_{\phi_{t}^{i}}E_{t}\left\{ \mu_{t}^{i}\phi_{t}^{i}+\nu_{t}^{i}\right\}
\end{split}
\end{equation*}

with the reframed incentive constraint

\begin{equation}
\theta\phi_{t}^{i}\leq(=)\psi_{t}^{i}=\mu_{t}^{i}\phi_{t}^{i}+\nu_{t}^{i}
\end{equation}

where

\begin{equation*}
R_{t+1}^{b,i}=\frac{(Z_{t+1}^{i}+Q_{t+1}^{i})}{Q_{t}^{i}} 
\end{equation*}

Then with the combined aggregate-idiosyncratic productivity process $\varPsi$
and aggregate sunspot process $\varXi$ the maximization can be written in a
discrete form as

\begin{eqnarray*}
\psi _{t}^{i} &=&\max_{\phi _{t}^{i}}\sum_{Z^{\prime },\xi agg^{\prime },\xi
i^{\prime }}\left( p(Z^{\prime agg^{\prime }},\xi ^{i^{\prime }}|Z,\xi
^{agg^{\prime }},\xi ^{i})\right)  \\
&&\left\{ \beta \left( 1-\sigma +\sigma \psi _{t+1}^{i}\left( S^{\prime
},\phi \right) \right) \left[ (R_{t+1}^{b,i}\left( S^{\prime },\phi \right)
-R_{t+1}^{i})\phi _{t}^{i}+R_{t+1}^{i}\right] \right\} 
\end{eqnarray*}

where for any state-shock combination resulting in a run the state value is
simply 0 and S' is the external state including the global distribution of
bank and household positions and the new shocks. The summation over the
idiosyncratic sunspot shocks may be collapsed into an expression for the
probability banks survive given the productivity shocks and aggregate
sunspot if the idiosyncratic sunspot shock's distribution is defined.

The household's problem is similarly adjusted. Beginning with:

\begin{equation*}
V_{t}^{h,i}=\max_{Kh,RD,g}E_{t}\left(u(C_{t}^{h,i})+V(.)\right) 
\end{equation*}

and the income = expenditure constraint:

\begin{eqnarray*}
&&C_{t}^{h,i}+D_{t}^{i}+Q_{t}^{i}K_{t}^{h,i}+f(K_{t}^{h,i})+p_{t}(g_{t+1}^{i}-w^{i}G)+h(g_{t+1}^{i},w^{i}G)
\\
&=&Z_{t}^{i}W^{h}w^{i}+R_{t}^{i}D_{t\text{\textminus}%
1}^{i}+(Z_{t}^{i}+Q_{t}^{i})K_{t\text{\textminus}1}^{h,i}+g_{t}^{i}
\end{eqnarray*}

It then becomes

\begin{eqnarray*}
V_{t}^{h,i} &=&\max_{Kh,RD,g}u(C_{t}^{h,i}\left( S,Kh,RD,g\right) ) \\
&&\sum_{Z^{\prime },\xi agg^{\prime },\xi i^{\prime }}\left( p(Z^{\prime
agg^{\prime }},\xi ^{i^{\prime }}|Z,\xi ^{agg^{\prime }1},\xi ^{i})\right)
\left( V(S^{\prime },Kh,RD,g)\right) 
\end{eqnarray*}

Similarly, the size of the summation can be reduced by breaking into
next-period run and no-run states, with probabilities of each arising (given
Z' and $\xi ^{agg}$') a function of the distribution of $\xi ^{i}$.

\textbf{\bigskip }

\textbf{Luke -- this is good progress. \ I think that you will want to work
to streamline the notation. \ \ Is the bottom line that a sunspot driven
bank run (a) affects rewards to the parties so that it has welfare
consequences; (b) affects various state variables going forward (so setting
off a version of the transition dynamics that you looked at previously) and
(c) that "markov sunspots" may lead to "expected bank runs."?}

\textbf{\bigskip }

\textit{I would say (a) and (b) are entirely accurate, and (c) is basically 
correct, but I'd put it that there's a continuum of expectedness over sunspot 
states given fundamental states, more than a straight expected/unexpected 
situation.}

\end{document}
