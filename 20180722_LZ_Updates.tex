\documentclass[english]{article}
\usepackage[T1]{fontenc}
\usepackage[latin9]{inputenc}
\usepackage{geometry}
\usepackage{graphicx}
\geometry{verbose,tmargin=1in,bmargin=1in,lmargin=1in,rmargin=1in}
\usepackage{textcomp}
\usepackage{amsmath}
\usepackage{babel}
\usepackage{float}
\begin{document}
\begin{verbatim}
TO:        Robert G. King
FROM:      Luke Zinnen
DATE:      22 July, 2018
SUBJECT:  Mid-summer updates
\end{verbatim}

\section{Overview}
This memo is intended to update the status of my work through mid July, principally focusing 
on the solution of the GK model.

\section{GK: Partial Equilibrium Testing}
Taking a snapshot of outer loop values (asset price, interest rate on deposits, bank net worth, and (unused in the inner loop) household 
consumption) from the existing solution method, I separately  perturbed each of the values by a uniform 
10bp and solved the bank's problem with both value and policy function iteration, comparing the implied 
policy functions and newly estimated values to estimates from the baseline input values. 

This yielded several results, including important distinctions between results for value and policy 
function iteration; a large degree of volatility in output estimates for many variables as small moves are 
made through the state space; consistent and large changes for others, often following the specific 
region of the state space near the positive density portions of the bank's ergotic distribution; and extreme 
sensitivity of the policy function to asset price increases in particular, with even increases <1bp 
resulting in degenerate ergotic distributions at the origin.

These observations have implied to me two requirements for attempts to solve the GK problem recursively 
going forward. First and more important, any method used must very accurately estimate each outer 
iteration/step backward in time, given the future: the high instability of estimates to small changes in 
inputs means that imperfections cannot easily be washed away over the course of outer iterations, so 
getting present values as close to correct as possible conditional on future values is necessary before 
advancing further back in time. This suggests a process not of value or policy function iteration, but 
repeated one-step solutions holding future values fixed and updating present estimates until they converge, 
only then taking a step backwards in time. 

Second, there is likely room for improvement by increasing the density of the state space, even if 
doing so alone is (per prior observations with simplified models in which an investment technology 
took the place of the banking sector) unlikely to do so on its own while maintaining feasible memory 
usage and computation time.

\section{GK: Subsequent Solution Attempts}
I began by implementing the above described method of repeatedly solving the bank's problem for a 
given period until it converges before moving back a time step, seeding it with the interim estimates 
used for the partial equilibrium testing. This resulted in non-convergence for even a single time step.

I followed by increasing the number of states as follows: retain 5 productivity levels; increase 
the capital levels to 151 from 51, and their spacing from $\{0.0^2, 0.02^2 ... 0.98^2, 1.0^2\}$ to 
a more uniform (but still denser at lower values) $\{0^{1.5}, (1/150)^{1.5} ... (149/150)^{1.5}, 1.0^{1.5}\}$; 
increase the repayment levels to 65 from 31, decrease their range from [0, 1] to [0, 0.75], and likewise 
shift to a more uniform $x^{3/2}$ spacing from $x^2$; and beginning from a newly calculated start based 
on permanently fixed productivity levels. 

This led to rapid (usually instant) convergence to new outer loop estimates, but little indication of 
convergence over time periods. Investigation indicated that some of this may have been due to deterioration 
of tools used with the increased variable size (notably, with variables >2GB in size requiring different 
memory allocation): first sparse methods and then basic BLAS *GEMM were shown to result in incorrect 
matrix multiplication (notably leaving many--and in the case of sparse, an impossible pattern of--incorrect 0s 
in continuation value calculations). This was corrected (and accelerated) by exploiting the structure of the 
relevant matrices, and the output of that specific operation confirmed as (at least for initial columns) accurate. 

Figures for basic method:
\begin{figure}[H]
\centering
		\includegraphics[width=.5\textwidth]{meanDiff.png}\includegraphics[width=.5\textwidth]{maxDiff.png}
\end{figure}
Console output for setting with trailing average:
\begin{figure}[H]
\centering
		\includegraphics[width=.5\textwidth]{_consoleOutput.png}

First line is maximum update distance relative to previous estimate (middle two output groups vertically, numbers in upper left 1 and 2) or previous period values (first and last, numbers 500 and 501), in order: Deposit rate, asset price, bank net worth, household consumption, recovery ratio, default probability. Third line is the same for average values.
\end{figure}


However, thus far period by period update sizes have not substantially improved: with the basic method 
of setting the new future value to the converged estimate, maximum and average update sizes remain 
generally in the neighborhood of $1.0-2.0^{-1}$ and mid-upper $10^{-3}$, respectively, or with instead setting the future values to the 50-period 
trailing average of estimates the estimate differences around half that. 



Looking at the time path of specific variable values shows a similar picture: asset price, for example, 
is relatively stable in the default state (oscillations of order 250bp against price levels of order 1) 
compared to the recovery state, the first state when bankers reenter after a bank run, (oscillations 
of order 1000bp against price levels of order 1). In both cases, the price series line up as expected, 
with higher prices the higher productivity levels are, and the average prices for each productivity level 
are higher in the recovery state than the run state, likewise as expected. However, there is a major reversal 
from all previous results, in that prices are substantially higher: a representative fixed point case close to the long-run
states in the GK paper (bank holds ~72\% of capital, funded with 0.68 debt) has a price of 1.0658 compared 
to the calibrated steady state cost of 1, and all prices are likewise shifted up relative to even perfect 
foresight paths, where I would expect that rather prices should be lower due to the possibility of 
adverse productivity shocks and (where applicable) potential bank runs. Since this is a new and surprising outcome 
compared to all previous trials, I would want to confirm that no other size-induced bugs are present before 
treating it as reliable.

\begin{figure}[H]
\centering
		\includegraphics[width=.5\textwidth]{Qrun.png}\includegraphics[width=.5\textwidth]{Qrecov.png}
\end{figure}

Traces of other variables demonstrate the high variance as well, in even more dramatic form: take, 
for example, the deposit rate in the recovery state, which not only shows no meaningful trend in level or 
dispersion, but shows little relation between productivity and interest rate, with all levels overlapping 
at random. The same is true of the expectation of the bank's net worth next period (closely related to 
that period's contribution to the state's value to the bank), conditioned on its chosen policy. There is 
a clear sense that there is not in general a high probability of default directly following a recovery, but 
that is all that is consistent.

\begin{figure}[H]
\centering
		\includegraphics[width=.5\textwidth]{Rrecov.png}\includegraphics[width=.5\textwidth]{ENrecov.png}
\end{figure}

\section{GK Solution: Next Steps}
Besides making sure new behavior is not caused by new bugs, there are a few possible lines of action 
that have not yet been explored yet. I have not yet tried increasing the fineness of the productivity 
state variable and transition matrix; allowing smaller gradations there (currently the spacing is at +/-14\%
 and +/- 7\% from the neutral level) and a smoother transition function could be useful where additional 
capital or debt levels may be seeing diminishing returns, especially relative to computational constraints. 
There is room yet for expansion of the state space, and adding productivity levels increases the size 
only linearly rather than quadratically (the largest components in working memory, as opposed to I/O, 
are the size of the number of no-run states times the number of policies in principle available, 
$numZ*(numKb*numRD)^2$). 

More runtime on its own could conceivably help as well; the input value is based on a guess and only 
slowly updates. Leaving aside state transitions, 100 time periods only reduces a period's contribution 
by about 2/3, but this would be worthwhile mainly after confirming the step by step process is working as 
intended.

Applying some sort of damping on the outer loop besides the 50-period moving average is unlikely to 
help, as I have tried that before in alternative solution methods. 

\subsection{Next Steps: Progress}
I have identified another feature in the solution code which was resulting in inaccurate computations, 
and additionally recoded the outer-loop algebra to tighten up calculations in cases where a bank would 
fundamentally default. The latter I do not expect to have had a very great effect, at least on the values
obtaining in states which are in practice traversed. Trials with updates show some initial promise in 
excess of the previous round, with at least some degree less of wild swings in values and bank run/early recovery asset prices
more in line with expectations: substantially below the original GK steady state value rather than above.

Step by step variance remains high, but while not encouraging it is premature to conclude the method
will be a failure, as mean values of key tracked variables (chiefly run and recovery state asset prices) 
have only just leveled out at the end of the most recent run of 3000 outer loop iterations, so active 
and meaningful movements were to be expected based purely on that, and may continue for some time.

\begin{figure}[H]
\centering
		\includegraphics[width=\textwidth]{_prev update generation 5.png}
\end{figure}

A parallel trial using 21 rather than 5 productivity levels (but reducing the number of capital and 
debt levels to try to keep computational speed and storage comparable) has been showing similar 
behavior. 

\begin{figure}[H]
\centering
		\includegraphics[width=\textwidth]{_last update generation 6.png}
\end{figure}

\section{Other Projects}
Active work on potential new projects has been limited. However, through the paper presentation of 
July 22, I have identified a possible direction, but without details well worked out yet. This would be work as described in a previous meeting 
relating to how much currencies are used as reserves, in a similar space to "Banks, Trade, and Dominant Currencies," 
focused more directly on central banks, governments, and local financial systems. I am in the process of 
collecting relevant data to inform what direction any further work on this project should take. 

Additionally, I have just begun thinking about how to construct a model to explain 1) the presence of 
banks of many sizes, ranging well over 2 orders of magnitude (under \$50 billion for ``small" banks 
up to over \$2 trillion for JPM), and 2) why for many measures small banks are more profitable (or 
less risky, with comparable return on equity but lower leverage). 

\end{document}                          