\documentclass[english]{article}
\usepackage[T1]{fontenc}
\usepackage[latin9]{inputenc}
\usepackage{geometry}
\geometry{verbose,tmargin=1in,bmargin=1in,lmargin=1in,rmargin=1in}
\usepackage{textcomp}
\usepackage{amsmath}
\usepackage{babel}
\begin{document}
\begin{verbatim}
TO:        Robert G. King
           Stephen J. Terry
FROM:      Luke Zinnen
DATE:      20 May, 2018
SUBJECT:   Bank Policies: Partial Equilibrium Testing
\end{verbatim}

\section{Overview}

This memo provides a proposal for near-term investigation of bank policy in my model. It is intended* 
to isolate features which may underly computational instability in the full general equilibium model. 
The exercise is to generate a partial equilibrium setting in which banks choose policies subject to 
their constraints and objectives, but do not influence prices. 

 

*In parallel with the exercise of creating a general equilibrium model with as simple a stand-in for 
the banking sector as feasible.

\section{Setup}
This exercise will be built off of results from the code for solving the model in general equilibrium.
Following an initialization procedure, the code will be allowed to run until price and quantity updates 
in the outer loop no longer become smaller with further iterations. At this point, a snapshot of the 
resulting prices and quantities (chiefly $Q$, the price of the capital asset; $R$, the risky interest 
rate paid on deposits; and $\pi^{live}$, the probability of a bank not experiencing a bank run in a 
given state) will be taken for use as the basis of the partial equilibrium exercise. 

These values will be used as inputs to the inner loop solving the bank's policy problem: given asset 
prices, interest rates, and default probabilities, to choose capital holdings and debt issuance to 
maximize the expected discounted value of payouts to continuing shareholders, subject to their balance 
sheet constraint (the value of their assets must be no greater than the sum of their equity and liabilities) 
and incentive compatibility constraint (the value of the bank as a going concern must equal or exceed 
the fraction of its liquidation value bankers can run away with). 


\section{Output and Tests}
Two methods of determining the bank's policy function will be used: policy function iteration and 
value function iteration. Although the results for both methods are expected to be the same in terms 
of the resulting policy and within the margin of the value function iteration's halting tolerance, 
this will serve as confirmation and inform whether they are similarly stable and their relative speeds 
of execution. 

The outputs will be the policy function and the prices, quantities, and probabilities it implies for 
the period prior. These values will be compared with their corresponding inputs to inform where 
the largest changes are coming from, and thus potentially the instability. Additionally, I will plot 
the ergotic distribution of the policy funciton, and compare it with both the implied values and the 
input-to-update changes state by state. 

Going beyond reporting the baseline using the snapshot directly, I will make small changes to the asset 
price, interest rate, and probability of default. Making one parallel change across all states, for 
one of these values, for the current or future period value or for both at once.* The differences between 
policy functions under the new prices and the baseline will be noted, as will induced changes in 
updated values. Again, the ergotic distribution will be reported, along with changes due to the new 
values defining the partial equilibrium setting. 

 

*Interest rates will be varied only for the current period, default probabilities only for the future, 
and asset prices for both. When future asset prices are changed, so will default probabilities, to 
correspond with the (currently linear) relationship between depositor recovery ratio and default 
probability. When future values are varied, however, current period values will remain fixed rather 
than follow the relationship implied by the outer loop (household and aggregation effects), as the 
main area of interest is currently the bank's decision rule.

\section{Results and Goals}
The results coming from the baseline exercise are intended to identify regions where there are rapid 
changes in values (for example, bank run probabilities). Especially if these rapidly changing values 
are associated with large changes in bank policy, they could be driving the non-convergence in computation. 
Comparison with the ergotic distribution will inform which regions are likely to be important in a 
different way: if a state is never traversed in equilibrium, it is unlikely to be of great importance 
whether or not its values change a great deal in the outer loop updating process (provided of course 
that the change is not one that would result in the state being traversed).

Results from varying values used to define the partial equilibrium serve two purposes. The first is 
to explore the sensitivity of the policy function (and resulting ergotic distribution) to changes in 
constraints or payoffs. In addition to showing the absolute degree of changes that result, the output 
will inform as to the relative strengths of changes for the different input values: asset price vs. 
interest rate vs. default probability, present vs. future, and in the case of varying the interest rate 
or current period asset price, cleanly separating effects due to varying constraints from varying 
payoffs. Additionally, if the ergotic distributions remain similar, that may indicate it is safe to 
weight large desired updates as unimportant in those regions not traversed.  Second is to determine 
the degree to which shifts in the values move the locations of high-gradient regions in the output, 
as that would be a likely driver of non-convergence. 

\end{document}
