\documentclass[english]{article}
\usepackage[T1]{fontenc}
\usepackage[latin9]{inputenc}
\usepackage{geometry}
\usepackage{graphicx}
\geometry{verbose,tmargin=1in,bmargin=1in,lmargin=1in,rmargin=1in}
\usepackage{textcomp}
\usepackage{amsmath}
\usepackage{babel}
\usepackage{float}
\begin{document}
\begin{verbatim}
TO:        Robert G. King
FROM:      Luke Zinnen
DATE:      07 Nov, 2018
SUBJECT:  GK Steady State with Idiosyncratic Productivities -- Probability Structure
\end{verbatim}

\section{Overview}
This discusses the probability structure used in my analysis of the steady-state of a GK extension in 
which there is a continuum of banks which receive different productivity levels each period, and make 
maximizing decisions conditional on their own transition probabilities. It covers the structure I have 
been using, conjectures regarding what will be necessary to keep it well behaved while scaling it up, 
and ends with a brief section concerning how the incentive compatibility constraint affects other features of the parameterization and 
equilibrium.

\section{The Structure Used So Far}
I have used two levels of productivity with transition probabilities as follows: there is a $\gamma$ 
probability of remaining at the same level, and a $(1-\gamma)$ probability of drawing a new level, which 
will be the high state with probability $p^h$. 

I chose this structure because the analytical investigation indicated that problems in finding equilibria 
arise mainly due to characteristics of the highest state: it is there that the potential for infinite-valued 
banks arises (or rather, for a probability structure in which the probability of being in higher states next period does not 
decrease with the level of the current state, the highest state will be the first to become explosively valued). 

Similarly, only one low-level state is necessary to allow defaults and interbank lending to take place. 

To begin, I choose an average level of the productivity of the capital in the economy, because jointly 
raising the whole productivity distribution (along with parameters I'd prefer to keep roughly constant 
relative to productivity: household capital management cost and endowments) would have no effect apart 
from increasing the price of capital by the same factor. Following GK, I choose approx. 0.0126. I then choose the fraction by which the high 
state's productivity should exceed the average and a probability of selecting the high state in the 
$(1-\gamma)$ fraction that a bank does not remain in its previous state. These two choices pin down 
the productivity level and probability of drawing the low state: higher productivity in the high state 
pushes down productivity in the low state, as does higher probability of the high state. $\gamma$, the 
autocorrelation of productivity, is independent of the other choices.

The equilibria of interest feature default when a bank transitions from high to low productivity. This 
requires that the rate of return on assets in the low state be insufficient to pay deposits: 
$\phi*(Z^l + Q)/Q < R^h*(\phi^h - 1)$, where $h$ denotes levels associated with high states and 
$l$ low states, $Z$ the productivity level, $Q$ the asset price, $R$ the deposit rate, and $\phi$ the 
leverage of the bank, $(\phi - 1)$ thus the fraction of bank assets funded by deposits. 

Consider initially the case of $Z^l=0$, or the net rate of return in the low state being 1. Then for 
a bank to exactly 0 equity (and thus not push up the rate on deposits) when the rate of time preference is $\beta = 0.99$, its leverage would have 
to be approximately 100. Increasing leverage from there results in losses and pushes up $R$, but 
relatively slowly, as it is linked to the probability of having a high to low transition. Likewise, 
even with infinite leverage the gross rate of return on deposits would be 1, so absent a very high 
(greater than 0.5) probability of the high to low transition, the interest rate would remain low, under 
about 1.0202 (which if taken as granted would require leverage of near 50 to result in default in the 
first place). Besides not wanting to calibrate to such high leverage, steady state leverage this high 
will imply very high market to book values of banks for reasonable loan spreads (GK calibrated to 0.0010): 

$\psi_h = [(spread_{hh})*\phi_h + R_h](1-\sigma+\sigma*\psi_h)[\gamma + (1-\gamma)(1-p_h)]$

and if $[(spread_{hh})*\phi_h + R_h]*[\gamma + (1-\gamma)(1-p_h)] > 1$ the value will be infinite as 
can be seen by an iterative process up updating $\psi$ holding the rest constant. With $\phi$ of 100, $R_h$ 
assumed still 1.01, and taking the GK spread as a low estimate of the spread given higher than average 
physical return per unit of assets, $[\gamma + (1-\gamma)(1-p_h)]$ must be under about 0.5, implying 
even with complete mean reversion ($\gamma = 0$) the physical return on capital in the high state has 
to be over 2 times the average to generate even as little as 0 net return in the bad state. 

Simply put, you can't get default in steady state without unreasonably high leverage unless there are 
actual losses in bad states, and even this level of productivity in the low state requires very high probability 
or very high productivity in the high state.

A sufficiently negative net return in the low state reduces the required leverage for default, but requires 
either raising the productivity of the high state or increasing its probability. With full mean reversion, this 
is not an insurmountable problem, as inspection has shown that $\gamma = 0$ with sufficiently high probability in the high state 
(0.99) and level of the high state ( > 0.66 times the average) gives results with the appropriate default structure  
and steady state aggregates comparable to GK. 

Moving away from IID, however, introduces parameterizations in which no equilibrium is found at all 
that conforms to a default structure desired, though sufficiently extreme parameterizations likewise can be found 
which will result in the desired survival outcomes, which have higher leverage, total capital held by banks, and price of capital.

What should be noted here is the high concentration at the high end required to get bad enough low states 
to generate defaults. There isn't much scope to make a richer distribution of states that have meaningful mass 
without spreading the high state. However, maintaining the average productivity means filling out the 
region between the two points requires increasing the productivity of some portion of the high distribution. 
If expected future productivity strictly increases with current productivity, then an increasing spread on loans 
on the highest end (which also loosens the leverage constraint) can be destabilizing, as behavior is dominated by 
the most extreme state. This will require not only the potential to fall to a somewhat less high state, but 
also larger expected losses in default.

For this reason, I believe that for a probability structure with significant persistence and a full 
support from lowest state to highest, a highly asymmetrical distribution is necessary with a long 
left tail and most of the probability mass close to the average.

\section{Effects of the IC Constraint on Equilibrium}
The IC constraint, $\theta\phi\leq\psi$ governs the degree of leverage a bank in a given state operates with.
As $\theta$ falls, leverage $\phi$ is allowed to increase, which in expectation also increases the market 
to book ratio of the bank $\psi$. The increased leverage at the bank will push up the amount of capital 
operated by banks collectively, and by the household's first order condition for capital push up the price 
of capital, reducing the loan spreads available to banks at all productivity levels, a general equilibrium force 
opposing the tendency of higher allowed leverage to increase $\psi$s. Likewise, the increase in 
leverage (weakly) increases expected losses on deposits, pushing up interest rates paid on deposits, reducing
the lending spread from the other end, and thus mitigates the increase in $\psi$ as well. 

An increase in bank system leverage could be accommodated without increasing the amount of capital operated 
by banks through increasing the rate of banker exit and/or reducing the endowment of new bankers.

A more weakly binding IC constraint is expected to increase the degree to which the highest productivity 
banks dominate an equilibrium with many productivity levels and persistence. This is because higher 
amounts of bank capital operated drive down returns and spreads for all banks, resulting in the cutoff 
point between which a bank prefers to invest in capital itself rather than lend in the interbank market 
to shift to a higher productivity level.







\end{document}                          